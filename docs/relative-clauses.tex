\documentclass[nols,twoside,nofonts,nobib,nohyper]{tufte-handout}

\usepackage{fixltx2e}
\usepackage{tikz-cd}
\usepackage{tcolorbox}
\usepackage{appendix}
\usepackage{listings}
\lstset{language=TeX,
       frame=single,
       basicstyle=\ttfamily,
       captionpos=b,
       tabsize=4,
  }

\begin{acronym}
\acro{sfa}{Scopal Function Application}
\acro{fa}{Function Application}
\acro{wco}{Weak Crossover}
\acro{ScoT}{Scope Transparency}
\end{acronym}

\renewcommand*{\acsfont}[1]{\textsc{#1}}

\makeatletter
% Paragraph indentation and separation for normal text
\renewcommand{\@tufte@reset@par}{%
  \setlength{\RaggedRightParindent}{0pt}%
  \setlength{\JustifyingParindent}{0pt}%
  \setlength{\parindent}{0pt}%
  \setlength{\parskip}{\baselineskip}%
}
\@tufte@reset@par

% Paragraph indentation and separation for marginal text
\renewcommand{\@tufte@margin@par}{%
  \setlength{\RaggedRightParindent}{0pt}%
  \setlength{\JustifyingParindent}{0pt}%
  \setlength{\parindent}{0pt}%
  \setlength{\parskip}{\baselineskip}%
}
\makeatother

\setcounter{secnumdepth}{3}

\title{...}

\author[Patrick D. Elliott]{Patrick~D. Elliott}

\addbibresource[location=remote]{/home/patrl/repos/bibliography/elliott_mybib.bib}

\lingset{
  belowexskip=0pt,
  aboveglftskip=0pt,
  belowglpreambleskip=0pt,
  belowpreambleskip=0pt,
  interpartskip=0pt,
  extraglskip=0pt,
  Everyex={\parskip=0pt}
}


% \usepackage{booktabs} % book-quality tables
% \usepackage{units}    % non-stacked fractions and better unit spacing
% \usepackage{lipsum}   % filler text
% \usepackage{fancyvrb} % extended verbatim environments
%   \fvset{fontsize=\normalsize}% default font size for fancy-verbatim environments

% % Standardize command font styles and environments
% \newcommand{\doccmd}[1]{\texttt{\textbackslash#1}}% command name -- adds backslash automatically
% \newcommand{\docopt}[1]{\ensuremath{\langle}\textrm{\textit{#1}}\ensuremath{\rangle}}% optional command argument
% \newcommand{\docarg}[1]{\textrm{\textit{#1}}}% (required) command argument
% \newcommand{\docenv}[1]{\textsf{#1}}% environment name
% \newcommand{\docpkg}[1]{\texttt{#1}}% package name
% \newcommand{\doccls}[1]{\texttt{#1}}% document class name
% \newcommand{\docclsopt}[1]{\texttt{#1}}% document class option name
% \newenvironment{docspec}{\begin{quote}\noindent}{\end{quote}}% command specification environment

\begin{document}

\maketitle% this prints the handout title, author, and date

We can assume that the relative operator is just an identity function:

\ex
$Op_{Rel} ≔ λ x . x$\hfill$\type{r → r}$
\xe

We can shift a relative operator into a scope taker via a shifter we can call $\ml{Q}_{Rel}$.

\ex
$Q_{Rel} m ≔ λ k . λ x . k (m x)$\hfill$\type{(r → a) → (a → b) → r → b}$
\xe

Now we have everything we need to get simple pied-piping:

\ex
Mary met the man [[the portrait of whom] John bought].
\xe

Let's start with the composition of the pied-piped DP -- we shift the relative
operator to a scope-taker via $Q_{Rel}$, and scope it to the edge:

\ex
\begin{forest}
  [{$λ x . ιy[y \ml{portrait} x]$}
  [{$λ k . λ x . k x$}
    [{$Q_{Rel}$}]
    [{$Op_{Rel}$}]
  ]
  [{$λ x . ιy[y \ml{portrait} x]$}
  [{$λ x$}]
  [{...}
    [{the}]
    [{...}
      [{portrait}]
      [{$t_{x}$}]
    ]
  ]
  ]]
\end{forest}
\xe

Now we shift the result into a scope-taker via $Q_{Rel}$, and scope it to the
edge of the relative clause, returning a derived predicate:

\ex
\begin{forest}
  [{$λ x . j \ml{bought} ιy[y \ml{portrait} x]$}
  [{$λ k . λ x . k (ι y[y \ml{portrait} x])$}
    [{$Q_{Rel}$}]
    [{$λ x . ιy[y \ml{portrait} x]$} [{the portrait of whom},roof]]
  ]
  [{$λ x . j \ml{bought} z$}
    [{$λ x$}]
    [{...}
      [{John}]
      [{...}
        [{bought}]
        [{$t_{x}$}]
      ]
    ]
  ]
  ]
\end{forest}
\xe


\printbibliography

\end{document}
