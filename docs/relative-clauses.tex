\documentclass[nols,twoside,nofonts,nobib,nohyper]{tufte-handout}

\usepackage{fixltx2e}
\usepackage{tikz-cd}
\usepackage{tcolorbox}
\usepackage{appendix}
\usepackage{listings}
\lstset{language=TeX,
       frame=single,
       basicstyle=\ttfamily,
       captionpos=b,
       tabsize=4,
  }

\begin{acronym}
\acro{sfa}{Scopal Function Application}
\acro{fa}{Function Application}
\acro{wco}{Weak Crossover}
\acro{ScoT}{Scope Transparency}
\end{acronym}

\renewcommand*{\acsfont}[1]{\textsc{#1}}

\makeatletter
% Paragraph indentation and separation for normal text
\renewcommand{\@tufte@reset@par}{%
  \setlength{\RaggedRightParindent}{0pt}%
  \setlength{\JustifyingParindent}{0pt}%
  \setlength{\parindent}{0pt}%
  \setlength{\parskip}{\baselineskip}%
}
\@tufte@reset@par

% Paragraph indentation and separation for marginal text
\renewcommand{\@tufte@margin@par}{%
  \setlength{\RaggedRightParindent}{0pt}%
  \setlength{\JustifyingParindent}{0pt}%
  \setlength{\parindent}{0pt}%
  \setlength{\parskip}{\baselineskip}%
}
\makeatother

\setcounter{secnumdepth}{3}

\title{Relative clauses}

\author[Patrick D. Elliott]{Patrick~D. Elliott}

\addbibresource[location=remote]{/home/patrl/repos/bibliography/elliott_mybib.bib}

\lingset{
  belowexskip=0pt,
  aboveglftskip=0pt,
  belowglpreambleskip=0pt,
  belowpreambleskip=0pt,
  interpartskip=0pt,
  extraglskip=0pt,
  Everyex={\parskip=0pt}
}


% \usepackage{booktabs} % book-quality tables
% \usepackage{units}    % non-stacked fractions and better unit spacing
% \usepackage{lipsum}   % filler text
% \usepackage{fancyvrb} % extended verbatim environments
%   \fvset{fontsize=\normalsize}% default font size for fancy-verbatim environments

% % Standardize command font styles and environments
% \newcommand{\doccmd}[1]{\texttt{\textbackslash#1}}% command name -- adds backslash automatically
% \newcommand{\docopt}[1]{\ensuremath{\langle}\textrm{\textit{#1}}\ensuremath{\rangle}}% optional command argument
% \newcommand{\docarg}[1]{\textrm{\textit{#1}}}% (required) command argument
% \newcommand{\docenv}[1]{\textsf{#1}}% environment name
% \newcommand{\docpkg}[1]{\texttt{#1}}% package name
% \newcommand{\doccls}[1]{\texttt{#1}}% document class name
% \newcommand{\docclsopt}[1]{\texttt{#1}}% document class option name
% \newenvironment{docspec}{\begin{quote}\noindent}{\end{quote}}% command specification environment

\begin{document}

\maketitle% this prints the handout title, author, and date

\todo[inline]{Do we face von Stechow's problem for pied-piping in relatives?}

\section{Null operator theory}

Let's start by defining a new type constructor:

\ex
$\type{R a ≔ e → a}$
\xe

The relative operators is an inhabitant of $\type{R e}$ (an identity function):

\ex
$Op_{Rel} ≔ λ x . x$\hfill$\type{R e}$
\xe

We can shift a relative operator into a scope taker via a shifter we can call
$\ml{Q}_{Rel}$.\sidenote{This is a flipped version of the \texttt{fmap} of the
\texttt{Reader} functor.}

\ex
$Q_{Rel} m ≔ λ k . λ x . k (m x)$\hfill$\type{R a → (a → b) → R b}$
\xe

Now we have everything we need to get simple pied-piping:

\ex
Mary met the man [[the portrait of whom] John bought].
\xe

\subsection{Composition via QR}

Let's start with the composition of the pied-piped DP -- we shift the relative
operator to a scope-taker via $Q_{Rel}$, and scope it to the edge:

\ex
\begin{forest}
  [{$λ x . ιy[y \ml{portrait} x]$}
  [{$λ k . λ x . k x$}
    [{$Q_{Rel}$}]
    [{$Op_{Rel}$}]
  ]
  [{$λ x . ιy[y \ml{portrait} x]$}
  [{$λ x$}]
  [{...}
    [{the}]
    [{...}
      [{portrait}]
      [{$t_{x}$}]
    ]
  ]
  ]]
\end{forest}
\xe

Now we shift the result into a scope-taker via $Q_{Rel}$, and scope it to the
edge of the relative clause, returning a derived predicate:

\ex
\begin{forest}
  [{$λ x . j \ml{bought} ιy[y \ml{portrait} x]$}
  [{$λ k . λ x . k (ι y[y \ml{portrait} x])$}
    [{$Q_{Rel}$}]
    [{$λ x . ιy[y \ml{portrait} x]$} [{the portrait of whom},roof]]
  ]
  [{$λ x . j \ml{bought} z$}
    [{$λ x$}]
    [{...}
      [{John}]
      [{...}
        [{bought}]
        [{$t_{x}$}]
      ]
    ]
  ]
  ]
\end{forest}
\xe

\subsection{Composition via (indexed) continuations}

Since continuations do meaning composition \textit{in-situ}, the semantic
component should be accompanied by a (directly compositional) syntactic
component.

I'll assume that the atoms of computations are pairs of phonological features
and meanings.

\newpage

\ex
Mary met the guy \fbox{the portrait of whom John bought}
\xe

\ex~
$Q_{Rel} (\mathbb{X},m) ≔ \left(\semtower{\mathbb{X} \ast []}{\emptyset}, \semtower{\lambda x\,.\,[]}{m\,x}\right)$
\xe

\ex~
$Op_{Rel}  ≔ \left(\textrm{whom},id \right)$
\xe

\ex~
$(m,n)^{↓}  \coloneq (m id, n id)$
\xe

\begin{footnotesize}
\begin{forest}
  [{$(\text{whom the portrait of $\emptyset$ John bought }\emptyset, λ x . j \ml{bought} ιy[y \ml{portrait} x])$}
  [{$↓$}]
  [{$\left(\semtower{\text{[whom the portrait of
              $\emptyset$]} ∗ []}{\text{John bought
            }∅}, \semtower\right, \semtower{λ x . []}{j \ml{bought} ιy[y \ml{portrait} x]}\right)$}
    [{John$^{↑}$}]
    [{$\left(\semtower{\text{[whom the portrait of
              $\emptyset$]} ∗ []}{\text{bought
            }∅}, \semtower\right, \semtower{λ x . []}{λ z . z \ml{bought} ιy[y \ml{portrait} x]}\right)$}
      [{bought$^{↑}$}]
      [{$\left(\semtower{\text{[whom the portrait of
              $\emptyset$]} ∗ []}{∅}, \semtower\right, \semtower{λ x . []}{ιy[y \ml{portrait} x]}\right)$}
        [{$Q_{Rel}$}]
        [{$(\text{whom the portrait of } \emptyset,λ x . ιy[y \ml{portrait} x])$}
        [{$↓$}]
        [{$\left(\semtower{\text{whom} \ast []}{\text{the portrait of
              }\emptyset},\semtower{λ x . []}{ιy[y \ml{portrait} x]}\right)$}
          [{the$^{\uparrow}$}]
          [{...}
            [{portrait$^{\uparrow}$}]
            [{...}
              [{of$^{\uparrow}$}]
              [{$\left(\semtower{\text{whom} \ast []}{\emptyset},\semtower{λ x . []}{x}\right)$}
                [{$Q_{Rel}$}]
                [{whom}]
              ]
            ]
          ]
        ]
      ]
    ]
  ]
  ]]
\end{forest}
\end{footnotesize}

We can think of the result of applying $Q_{Rel}$ as a tower, in
\citeauthor{barkerShan2015}'s (\citeyear{barkerShan2015}) terms.

\ex
$Q_{Rel} Q_{Op} = \semtower{λ x . []}{x}$\hfill$\type{\tower{R b}{b}{e}}$
\xe

Composition may now proceed \textit{in-situ} via \textit{lift} and
\textit{scopal function application}.

\ex
That John bought.
\xe

\begin{forest}
  [{$λ x . \ml{j buy} x$}
  [{$\downarrow$}]
  [{$\semtower{λ x . []}{\ml{j buy }x}$}
    [{John$^{↑}$}]
    [{...}
      [{$\semtower{[]}{\ml{buy}}$\\bought$^{↑}$}]
      [{$\semtower{λ x . []}{x}$} [{$Q_{Rel} Op_{Rel}$},roof]]
    ]
    ]]
\end{forest}
%
\begin{forest}
  [{$\type{R t}$}
  [{$\downarrow$}]
  [{$\type{\tower{R t}{t}{t}}$}
    [{$\type{\semtower{R t}{e}}$}]
    [{$\type{\tower{R t}{t}{e → t}}$}
      [{$\type{\semtower{R t}{e → e → t}}$}]
      [{$\type{\tower{R t}{t}{e}}$}]
    ]
    ]]
\end{forest}

\ex
The portrait of whom John bought.
\xe

\begin{forest}
  [{$λx . \ml{j bought} ιy[y \ml{portrait} x]$}
    [{$↓$}]
    [{$\semtower{λx . []}{\ml{j bought} ιy[y \ml{portrait} x]}$}
      [{John$^{↑}$}]
      [{...}
        [{bought$^{↑}$}]
        [{$\semtower{λ x . []}{ι y[y \ml{portrait} x]}$}
      [{$Q_{Rel}$}]
      [{$λ x . ιy[y \ml{portrait} x]$}
      [{$↓$}]
      [{$\semtower{λ x . []}{ι y[y \ml{portrait} x]}$}
        [{the$^{↑}$}]
        [{...}
          [{portrait$^{\uparrow}$}]
          [{$\semtower{λ x . []}{x}$} [{$Q_{Rel}$ $Op_{Rel}$},roof]]
        ]
      ]
    ]
    ]
  ]]]
\end{forest}
%
\begin{forest}
  [{$\type{R t}$}
    [{$↓$}]
    [{$\type{\tower{R r}{r}{t}}$}
      [{$\type{\semtower{R r}{e}}$}]
      [{$\type{\tower{R r}{r}{e → t}}$}
        [{$\type{\semtower{R r}{e → e → t}}$}]
        [{$\type{\tower{R r}{r}{e}}$}
      [{$Q_{Rel}$}]
      [{$\type{R e}$}
      [{$↓$}]
      [{$\type{\tower{R r}{r}{e}}$}
        [{$\type{\semtower{R r}{(e → t) → e}}$}]
        [{$\type{\tower{R r}{r}{e → t}}$}
          [{$\type{\semtower{R r}{e → e → t}}$}]
          [{$\type{\tower{R r}{r}{e}}$}]
        ]
      ]
    ]
    ]
  ]]]
\end{forest}

\section{Islands}

We can cash out the islandhood of relative clause with the following condition:

\ex
Islandhood of relative clauses\\
Relative clauses must denote an evaluated type ($\type{R e}$).
\xe

In a framework in which continuations are used for meaning composition, a
consequence will be that relative clauses with multiple relative operators won't compose.

\printbibliography

\end{document}
